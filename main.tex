\documentclass[a4paper]{article}

\usepackage{./schooltest}

% Demonstration
\usepackage{tikz,pgfplots}
\usepackage{siunitx}

\begin{document}

% --------
% CHANGES:
% --------

% 1) Custom question environments!
% 2) Personally, I think any large expressions in the questions would be better typeset as inline displaystyle.
%    This gives them a looser, larger look, which I think works.
%    The questions are somewhat standalone, as opposed to being part of a text paragraph, so they
%    can afford to be a bit larger than a normal line.
% 3) \left and \right removed for small brackets.
%    These macros change the spacing around delimiters.
% 4) Better symbol for real numbers. (depends on: amssymb)
% 6) Replaced literal ':' with \colon. (depends on amsmath)
% 5) Unnecessary: added a tiny bit of horizontal spacing before the \backslash.

\begin{question}{1}
    $1 + 1$ is:
    \begin{multiplechoice}
        \item 1
        \item 2
        \item 0
    \end{multiplechoice}
\end{question}


\begin{question}{7}
    Let $\displaystyle
    g\colon \mathbb{R}\hspace{0.05em}\backslash \{2\}\to \mathbb{R},~ g(x) = \frac{4}{(x-2)^{2}} - 1$
    \begin{subquestion}{\undefined}
        What is $g(x)$ if:
        \begin{subsubquestion}{1}
            $g(x) = 0$
            \shortanswer
        \end{subsubquestion}
        \begin{subsubquestion}{2}
            $\displaystyle
                g'(x) = 0
            $
            \longanswer{3}
        \end{subsubquestion}
    \end{subquestion}
    \begin{subquestion}{1}
        Is $g$ bijective?
        \begin{multiplechoice}
        \item Yes
        \item No
        \end{multiplechoice}
    \end{subquestion}
    \begin{subquestion}{1}
        Here's a free mark for using \LaTeX !
    \end{subquestion}
    \begin{subquestion}{2}
        What is the range of $g$?
        \begin{multiplechoice}
            \item $\mathbb{R}\hspace{0.05em}\backslash \{2\}$
            \item $\mathbb{R}$
            \item $\varnothing$
            \item $\{1, 2, 3, 4, 5\}$
            \item $\left\{x\colon -1 < x,~ x \in \mathbb{R}\right\}$
        \end{multiplechoice}
    \end{subquestion}
\end{question}

\begin{question}{1}
    \begin{minipage}[t]{0.47\linewidth}
    \begin{flushleft}
        \vspace{0.05cm}
        Sketch the graph of: $\displaystyle y=e^{x}$
        \longanswer{9}
    \end{flushleft}
    \end{minipage}\hfill%
    \begin{minipage}[t]{0.478\linewidth}
        \begin{flushright}
        \fbox{\begin{tikzpicture}
            \begin{axis}[
                axis x line=center,
                axis y line=center,
                width=6.5cm, height=6.5cm,
                scale only axis,
                xtick=false,
                ytick=false,
                xlabel=x,
                ylabel=y,
                x label style={at={(axis description cs: 0.95,0.48)},anchor=north},
                y label style={at={(axis description cs: 0.45,0.91)},anchor=south},
                xmin=-10, xmax=10,
                ymin=-10, ymax=10]
            \end{axis}
        \end{tikzpicture}}
        \end{flushright}
    \end{minipage}
\end{question}

\newpage

\begin{question}{10}
An ant is walking along a continuously \textit{stretching} piece of elastic at a speed of $\log_{e}\!1\, \si{cm\per\second}$. The elastic starts out at a length of $\log_{e-1}\!1\,\si{cm}$, and is \textit{stretched} such that its length increases at a constant rate of $\log_{e-1}\!1\, \si{cm\per\second}$. How long does it take for the ant to reach the end of the elastic?
\longanswer{28}
\end{question}
\end{document}
